\section{Prediction Step}

The prediction step in a pressure-based Navier-Stokes solver estimates the intermediate velocity field before applying the pressure correction. This step advances the momentum equations without enforcing the incompressibility constraint.

\subsection{Explicit Schemes}

Explicit time integration schemes compute the velocity update using only known values from the current time step. For instance, the Forward Euler method updates the velocity as
\begin{equation}
\mathbf{u}^{n+1} = \mathbf{u}^n + \Delta t \left( - (\mathbf{u}^n \cdot \nabla) \mathbf{u}^n + \nu \nabla^2 \mathbf{u}^n \right),
\end{equation}
where $\mathbf{u} = (u,v)$ is the velocity vector, $\nu$ is the kinematic viscosity, and $\Delta t$ is the time step size.

In the provided code, this update is implemented as:
\begin{verbatim}
u_new[1:-1, 1:-1] = u[1:-1, 1:-1] + dt * dudt1[1:-1, 1:-1]
v_new[1:-1, 1:-1] = v[1:-1, 1:-1] + dt * dvdt1[1:-1, 1:-1]
\end{verbatim}
Here, \texttt{dudt1} and \texttt{dvdt1} contain the explicit evaluation of the right-hand side terms (advection and diffusion) at time step $n$.

\subsection{Semi-implicit Schemes}

Semi-implicit schemes treat some terms explicitly and others implicitly to improve numerical stability without the full cost of solving a global linear system. Typically, the nonlinear advection terms are treated explicitly, while the linear diffusion terms are treated implicitly.

\subsubsection{Explicit Advection and Implicit Diffusion}

This approach discretizes the momentum equation as
\begin{equation}
\frac{\mathbf{u}^{n+1} - \mathbf{u}^n}{\Delta t} = - \left( (\mathbf{u} \cdot \nabla) \mathbf{u} \right)^n + \nu \nabla^2 \mathbf{u}^{n+1}.
\end{equation}
The diffusion term $\nu \nabla^2 \mathbf{u}^{n+1}$ is treated implicitly, while the advection term is evaluated explicitly at time $n$.

For the $u$-component of velocity at grid point $(i,j)$, using a finite difference discretization on a uniform Cartesian grid with spacings $\Delta x$ and $\Delta y$, the update formula reads
\begin{equation}
\begin{aligned}
u_{i,j}^{n+1} = \frac{
u_{i,j}^n + \Delta t \left[
- u_{i,j}^n \frac{u_{i,j}^n - u_{i,j-1}^n}{\Delta x}
- v_{i,j}^n \frac{u_{i,j}^n - u_{i-1,j}^n}{\Delta y}
\right] + \Delta t \nu \left(
\frac{u_{i+1,j}^{n+1} + u_{i-1,j}^{n+1}}{\Delta x^2} + \frac{u_{i,j+1}^{n+1} + u_{i,j-1}^{n+1}}{\Delta y^2}
\right)
}{
1 + 2 \Delta t \nu \left( \frac{1}{\Delta x^2} + \frac{1}{\Delta y^2} \right)
}.
\end{aligned}
\end{equation}

This formula corresponds to a Jacobi iteration for the implicit diffusion step, avoiding the need to solve a global linear system explicitly.

In code, this update is implemented as:
\begin{verbatim}
u_new[1:-1, 1:-1] = (
    u[1:-1, 1:-1] + dt * adv_u +
    dt * nu * ((u[1:-1, 2:] + u[1:-1, 0:-2]) / dx**2 +
               (u[2:, 1:-1] + u[0:-2, 1:-1]) / dy**2)
) / (1 + 2 * dt * nu * (1/dx**2 + 1/dy**2))
\end{verbatim}
where \texttt{adv\_u} contains the explicit advection terms evaluated at time $n$.

This semi-implicit scheme allows for larger stable time steps compared to fully explicit methods while remaining computationally efficient and relatively straightforward to implement.
