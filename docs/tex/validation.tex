\section{Validation}

\subsection{Taylor-Green Vortex}
The Taylor-Green Vortex (TGV) is a canonical benchmark problem for validating numerical solvers of the incompressible Navier-Stokes equations due to its exact analytical solution.
This solution describes a decaying vortex flow in a periodic domain, allowing for direct comparison between simulated and analytical results, given by:
\[
u(x, y, t) = \cos(\pi x) \sin(\pi y) \exp(-2 \nu t \pi^2),
\]
\[
v(x, y, t) = \cos(\pi y) \sin(\pi x) \exp(-2 \nu t \pi^2),
\]
\[
p(x, y, t) = -\frac{1}{4} \left( \cos(2\pi x) + \cos(2\pi y) \right) \exp(-4 \nu t \pi^2),
\]
where \( u \) and \( v \) are the velocity components in the \( x \)- and \( y \)-directions, respectively, \( p \) is the pressure, \( \nu \) is the kinematic viscosity, and \( t \) is the time.
The factor \( \pi^2 \) in the exponential decay term accounts for the specific domain scaling used in the implementation.

Validation is performed by computing the kinetic energy of the simulated flow and comparing it with the analytical kinetic energy at a given time, to obtain a quantitative measure of the solver's accuracy.
Periodic boundary conditions are applied to maintain the flow's continuity across the domain boundaries.

\subsection{Lid-Driven Cavity}

The Lid-Driven Cavity flow is another benchmark for validating numerical solvers, particularly for assessing the handling of boundary conditions and steady-state flow features in a confined domain.
Unlike the Taylor-Green Vortex, it lacks an analytical solution, so validation relies on comparison with established reference data from literature or prior simulations.

The problem setup involves a square domain of unit length with no-slip boundary conditions on three walls (\( u = v = 0 \)) and a moving top wall (lid) with a constant tangential velocity (\( u = 1, v = 0 \)).
The flow is driven by the lid motion, resulting in a primary recirculating vortex and, at higher Reynolds numbers, secondary vortices in the corners.
We use Neumann boundary conditions for pressure (\( \frac{\partial p}{\partial n} = 0 \)) at the walls.

Here, validation focuses on the stream function's minimum value, which is indicative of the primary vortex strength in the cavity.
According to Logg et al. (\href{https://doi.org/10.1007/978-3-642-23099-8}{2012}) this value after $2.5$s is $-0.061076605$.
The comparison with a reference value allows for an assessment of numerical accuracy and convergence towards expected flow behavior.
